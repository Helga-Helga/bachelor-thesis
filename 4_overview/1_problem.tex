\section{Постановка задачи.}
Есть два облака точек: исходное $S$ (sourсe) и целевое $T$ (target).
К точкам исходного множества $ \vec{s} \in S$ применили поворот $R \in \mathbb{R}^{3 \times 3}$
и перемещение $ \vec{b} \in \mathbb{R}^3$,
а также в процессе сканирования появился аддитивный гауссовый шум
с неизвестной дисперсией
\begin{equation}
  \vec{k}_s = R \cdot \vec{s} + \vec{b} + \vec{ \xi }_s, \qquad
  \vec{ \xi }_s \sim N \left( \vec{0}, \sigma^2 \cdot I \right)
\end{equation}
где $k \, : \, S \to T$~---~разметка, то есть функция,
которая сопоставляет каждой точке из исходного множества точку из целевого множества.

Задача состоит в таком выборе матрицы $R$ и вектора $ \vec{b}$,
при которых расстояние между множествами $ \vec{k}_s$ и $T$ было бы наименьшим.
В том случае, когда $S$ и $T$ --- конечны,
можно воспользоваться обычным методом наименьших квадратов.

Воспользовавшись методом максимального правдоподобия, получаем оптимизационую задачу
\begin{equation}
  P \left(k, R, \vec{b} \right) =
    \prod \limits_{ \vec{s} \in S}
      \frac{1}{ \sqrt{2 \pi \sigma^2}} \cdot
      e^{- \frac{ \left \Vert \vec{k}_s - R \cdot \vec{s} - \vec{b} \right \Vert^2}{2 \sigma^2}}
    \to \max \limits_{k, R, \vec{b}}.
\end{equation}
После логарифмирования,
умножения на константу и отбрасывания постоянных слагаемых и множителей получаем задачу минимизации
\begin{equation}
 E \left( k, R, b \right) =
 \sum \limits_{ \vec{s} \in S}
 \left \Vert \vec{k}_s - R \cdot \vec{s} - \vec{b} \right \Vert^2 \to \min \limits_{k, R, b}.
\end{equation}
Два множества зачастую не имеют взаимно-однозначного отображения.
Это может приводить к тому,
что два отрезка, расположенные под прямым углом в одном облаке точек,
могут соответствовать трём отрезкам в другом облаке точек,
между соседними парами которых углы по 45 градусов.
