  \section{Постановка задачи.}
Есть два множества:
исходное $S \subset \mathbb{R}^3$ (sourсe) и целевое $T \subset \mathbb{R}^3$ (target).
К точкам исходного множества $ \vec{s} \in S$ применили
поворот $R \in \mathbb{R}^{3 \times 3}, \, \det{R} = 1$
и перемещение $ \vec{b} \in \mathbb{R}^3$,
а также в процессе сканирования появился аддитивный гауссовский шум
с неизвестной дисперсией
\begin{equation*}
  \vec{k}_s = R \cdot \vec{s} + \vec{b} + \vec{ \xi }_s, \qquad
  \vec{ \xi }_s \sim N \left( \vec{0}, \sigma^2 \cdot I \right)
\end{equation*}
где $k \, : \, S \to T$~---~разметка, то есть функция,
которая сопоставляет каждой точке из исходного множества точку из целевого множества.

Задача состоит в таком выборе матрицы $R$ и вектора $ \vec{b}$,
при которых расстояние между $ \vec{k}_s$ и $R \cdot \vec{s} + \vec{b}$ для всех $ \vec{s} \in S$
было бы наименьшим.

В том случае, когда $S$ и $T$ конечны,
можно воспользоваться обычным методом наименьших квадратов

\begin{equation}\label{eq:least:squares}
 E \left( k, R, \vec{b} \right) =
 \sum \limits_{s \in S}
 \left \Vert \vec{k}_s - R \cdot \vec{s} - \vec{b} \right \Vert^2 \to \min \limits_{k, R, b}.
\end{equation}

Сумма квадратов отклонений между векторами~---~это то же самое,
что сумма квадратов отклонений между проекциями по каждой координате
\begin{equation*}
  E \left( k, R, \vec{b} \right) =
    E_x \left( k, R, \vec{b} \right) + E_y \left( k, R, \vec{b} \right) +
    E_z \left( k, R, \vec{b} \right) \to
    \min \limits_{k, R, b}.
\end{equation*}

Найдём, чему равна проекция произведения матрицы $R$ на вектор $s$ на все координаты
\begin{equation*}
  R \cdot \vec{s} =
  \begin{bmatrix}
    r_{xx} & r_{xy} & r_{xz} \\
    r_{yx} & r_{yy} & r_{yz} \\
    r_{zx} & r_{zy} & r_{zz}
  \end{bmatrix} \cdot
  \begin{bmatrix}
    s_x \\
    s_y \\
    s_z
  \end{bmatrix} =
  \begin{bmatrix}
    r_{xx} \cdot s_x + r_{xy} \cdot s_y + r_{xz} \cdot s_z \\
    r_{yx} \cdot s_x + r_{yy} \cdot s_y + r_{yz} \cdot s_z \\
    r_{zx} \cdot s_x + r_{zy} \cdot s_y + r_{zz} \cdot s_z
  \end{bmatrix} =
  \begin{bmatrix}
    \vec{r}_x \cdot \vec{s} \\
    \vec{r}_y \cdot \vec{s} \\
    \vec{r}_z \cdot \vec{s}
  \end{bmatrix},
\end{equation*}
где первая строка получившегося вектора~---~проекция $R \cdot \vec{s}$ на ось $x$,
вторая строка~---~проекция на ось $y$, третья~---~на ось $z$.

Распишем сумму квадратов отклонений через проекции
\begin{gather*}
  E \left( k, R, \vec{b} \right) =
  \sum \limits_{s \in S} \left(
    \vec{r}_x \cdot \vec{s} + \vec{r}_y \cdot \vec{s} + \vec{r}_z \cdot \vec{s} +
    b_x + b_y + b_z - k_{s_x} - k_{s_y} - k_{s_z} \right)^2 = \\
  = \sum \limits_{s \in S} \left[
    \left( \vec{r}_x \cdot \vec{s} + b_x - k_{s_x} \right) +
    \left( \vec{r}_y \cdot \vec{s} + b_y - k_{s_y} \right) +
    \left( \vec{r}_z \cdot \vec{s} + b_z - k_{s_z} \right) \right]^2 = \\
  = \sum \limits_{s \in S} \left( \vec{r}_x \cdot \vec{s} + b_x - k_{s_x} \right)^2 +
  \sum \limits_{s \in S} \left( \vec{r}_y \cdot \vec{s} + b_y - k_{s_y} \right)^2 +
  \sum \limits_{s \in S} \left( \vec{r}_z \cdot \vec{s} + b_z - k_{s_z} \right)^2.
\end{gather*}
Множество параметров, которые входят в каждую из трёх сумм, разные.
Тогда можем минимизировать проекции суммы квадратов отклонений на все координаты отдельно
\begin{gather*}
  \begin{cases}
    E_x = \sum \limits_{s \in S} \left( \vec{r}_x \cdot \vec{s} + b_x - k_{s_x} \right)^2 \to
    \min \limits_{r_x, b_x}, \\
    E_y = \sum \limits_{s \in S} \left( \vec{r}_y \cdot \vec{s} + b_y - k_{s_y} \right)^2 \to
    \min \limits_{r_y, b_y}, \\
    E_z = \sum \limits_{s \in S} \left( \vec{r}_z \cdot \vec{s} + b_z - k_{s_z} \right)^2 \to
    \min \limits_{r_z, b_z}.
  \end{cases}
\end{gather*}

Имеем линейную функциию, которая возводится в квадрат.
Это выпуклая фунция.
Значит, можно взять частные производные по $r_i$ и $x_i$ для всех $i \in \left\{ x, y, z \right\} $
и приравнять их к нулю.
Получим 4 уравнения для каждой координаты.
Запишем для $E_x$, для остальных~---~аналогично
\begin{gather*}
  \begin{cases}
    \frac{ \partial E_x}{ \partial b_x} =
    \sum \limits_{s \in S} 2 \left( \vec{r_x} \cdot \vec{s} + b_x - k_{s_x} \right ) = 0, \\
    \frac{ \partial E_x}{ \partial r_{xx}} =
    \sum \limits_{s \in S}
      2 \left( \vec{r_x} \cdot \vec{s} + b_x - k_{s_x} \right ) \cdot s_x = 0, \\
    \frac{ \partial E_x}{ \partial r_{xy}} =
    \sum \limits_{s \in S}
      2 \left( \vec{r_x} \cdot \vec{s} + b_x - k_{s_x} \right ) \cdot s_y = 0, \\
    \frac{ \partial E_x}{ \partial r_{xz}} =
    \sum \limits_{s \in S} 2 \left( \vec{r_x} \cdot \vec{s} + b_x - k_{s_x} \right ) \cdot s_z = 0.
  \end{cases}
\end{gather*}

Решаем первое уравнение относительно $b_x$.
Получаем
\begin{equation*}
  \sum \limits_{s \in S} b_x =
  \sum \limits_{s \in S} \left(k_{s_x} - \vec{r_x} \cdot \vec{s} \right ).
\end{equation*}
Слева получили сумму одинаковых слагаемых
\begin{equation*}
  \left| S \right| \cdot b_x + \vec{r}_x \sum \limits_{s \in S} \vec{s} =
  \sum \limits_{s \in S} k_{s_x}.
\end{equation*}
Распишем скалярное произведение
\begin{equation*}
  \left| S \right| \cdot b_x + \sum \limits_{s \in S} r_{xx} \cdot s_x +
  \sum \limits_{s \in S} r_{xy} \cdot s_y + \sum \limits_{s \in S} r_{xz} \cdot s_z  =
  \sum \limits_{s \in S} k_{s_x}.
\end{equation*}

Решаем остальные уравнения относительно $r_{xi}$ для $i \in \left\{ x, y, z \right\} $.
Видим, что для разных $i$ производная по $r_{xi}$ одинаковая,
потому находим решение для $r_{xx}$, а для остальных решение будет аналогичным
\begin{equation*}
  \vec{r}_x \sum \limits_{s \in S} \vec{s} \cdot s_x =
  \sum \limits_{s \in S} \left( k_{s_x} - b_x \right) \cdot s_x.
\end{equation*}
Распишем скалярное произведение
\begin{equation*}
  \sum \limits_{s \in S} r_{xx} \cdot s_x^2 + \sum \limits_{s \in S} r_{xy} \cdot s_x \cdot s_y +
  \sum \limits_{s \in S} r_{xz} \cdot s_x \cdot s_z + \sum \limits_{s \in S} b_x \cdot s_x =
  \sum \limits_{s \in S} k_{s_x} \cdot s_x.
\end{equation*}
Получаем систему уравнений
\begin{equation*}
  \begin{cases}
    \left| S \right| \cdot b_x + \sum \limits_{s \in S} r_{xx} \cdot s_x +
    \sum \limits_{s \in S} r_{xy} \cdot s_y + \sum \limits_{s \in S} r_{xz} \cdot s_z  =
    \sum \limits_{s \in S} k_{s_x}, \\
    \sum \limits_{s \in S} r_{xx} \cdot s_x^2 + \sum \limits_{s \in S} r_{xy} \cdot s_x \cdot s_y +
    \sum \limits_{s \in S} r_{xz} \cdot s_x \cdot s_z + \sum \limits_{s \in S} b_x \cdot s_x =
    \sum \limits_{s \in S} k_{s_x} \cdot s_x, \\
    \sum \limits_{s \in S} r_{xx} \cdot s_x \cdot s_y + \sum \limits_{s \in S} r_{xy} \cdot s_y^2 +
    \sum \limits_{s \in S} r_{xz} \cdot s_y \cdot s_z + \sum \limits_{s \in S} b_x \cdot s_y =
    \sum \limits_{s \in S} k_{s_x} \cdot s_y, \\
    \sum \limits_{s \in S} r_{xx} \cdot s_x \cdot s_z +
    \sum \limits_{s \in S} r_{xy} \cdot s_y \cdot s_z +
    \sum \limits_{s \in S} r_{xz} \cdot s_z^2 + \sum \limits_{s \in S} b_x \cdot s_z =
    \sum \limits_{s \in S} k_{s_x} \cdot s_z.
  \end{cases}
\end{equation*}
Запишем её в матричном виде
\begin{equation*}
  \begin{bmatrix}
    \left| S \right| & \sum \limits_{s \in S} s_x & \sum \limits_{s \in S} s_y & \sum \limits_{s \in S} s_z \\
    \sum \limits_{s \in S} s_x & \sum \limits_{s \in S} s_x^2 & \sum \limits_{s \in S} s_x \cdot s_y & \sum \limits_{s \in S} s_x \cdot s_z \\
    \sum \limits_{s \in S} s_y & \sum \limits_{s \in S} s_x \cdot s_y & \sum \limits_{s \in S} s_y^2 & \sum \limits_{s \in S} s_y \cdot s_z \\
    \sum \limits_{s \in S} s_z & \sum \limits_{s \in S} s_x \cdot s_z & \sum \limits_{s \in S} s_y \cdot s_z & \sum \limits_{s \in S} s_z^2
  \end{bmatrix}
  \begin{bmatrix}
    b_x \\
    r_{xx} \\
    r_{xy} \\
    r_{xz}
  \end{bmatrix} =
  \begin{bmatrix}
    \sum \limits_{s \in S} k_{s_x} \\
    \sum \limits_{s \in S} k_{s_x} \cdot s_x \\
    \sum \limits_{s \in S} k_{s_x} \cdot s_y \\
    \sum \limits_{s \in S} k_{s_x} \cdot s_z
  \end{bmatrix}.
\end{equation*}
Введём обозначения:
\begin{gather*}
  \sum \limits_{s \in S} s_i = S_i, \qquad i \in \left\{x, y, z \right\}, \\
  \sum \limits_{s \in S} s_i s_j = S_{ij}, \qquad i, j \in \left\{ x, y, z \right\}, \\
  \sum \limits_{s \in S} k_{s_x} = k, \\
  \sum \limits_{s \in S} k_{s_x} \cdot s_i = k_i, \qquad i \in \left\{ x, y, z \right\}.
\end{gather*}
Уравнение приняло следующий вид
\begin{equation*}
  \begin{bmatrix}
    \left| S \right| & S_x & S_y & S_z \\
    S_x & S_{xx} & S_{xy} & S_{xz} \\
    S_y & S_{xy} & S_{yy} & S_{yz} \\
    S_z & S_{xz} & S_{yz} & S_{zz}
  \end{bmatrix}
  \begin{bmatrix}
    b_x \\
    r_{xx} \\
    r_{xy} \\
    r_{xz}
  \end{bmatrix} =
  \begin{bmatrix}
    k \\
    k_x \\
    k_y \\
    k_z
  \end{bmatrix}.
\end{equation*}
Используем метод Крамера для решения системы линейных уранений.
Определитель $ \Delta $
\begin{gather*}
  \Delta =
  \left| S \right| \cdot S_{xx} \cdot S_{yy} \cdot S_{zz} -
  \sum \limits_{i \in \left\{ x, y, z \right\} } L_i +
  2 \cdot \left| S \right| \cdot S_{xy} \cdot S_{xz} \cdot S_{yz} - \\
  - \sum \limits_{i, j, k \in \left\{ x, y, z \right\} } L_{ijk} +
  2 \sum \limits_{i, j \in \left\{ x, y, z \right\} } L_{ij},
\end{gather*}
где введены обозначения при $i, j, k \in \left\{ x, y, z \right\}, \, i \neq j \neq k$
\begin{gather*}
  L_i = S_{jk} \cdot \left( \left| S \right| \cdot S_{ii} - S_i^2 \right), \\
  L_{ij} = S_i \cdot S_j \cdot \left( S_{ij} \cdot S_k - S_{ik} \cdot S_{jk} \right), \\
  L_{ijk} = S_i^2 \cdot S_{jj} \cdot S_{kk}.
\end{gather*}
Определитель $ \Delta_{b}$
\begin{gather*}
  \Delta_b =
  k \cdot S_{xx} \cdot S_{yy} \cdot S_{zz} - \sum \limits_{i \in \left\{ x, y, z \right\} } L_i^b +
  2 \cdot S_{xy} \cdot S_{xz} \cdot S_{yz} -
  \sum \limits_{i, j, k \in \left\{ x, y, z \right\} } L_{ijk}^b + \\
  + \sum \limits_{i, j \in \left\{ x, y, z \right\} } L_{ij}^b +
  \sum \limits_{i, j \in \left\{ x, y, z \right\} } \left( L_{ij}^b \right)',
\end{gather*}
где введены обозначения
\begin{gather*}
  L_{ij}^b = S_{ij}^2 \cdot S_{k} \cdot k_k, \\
  L_{ijk}^b = S_i \cdot S_{jj} \cdot S_{kk}, \\
  \left( L_{ij}^b \right)' = \left(
  S_i \cdot k_j + S_k \cdot k_i \right) \cdot \left( S_{ij} \cdot S_{kk} -
  S_{jk} \cdot S_{ik} \right)
\end{gather*}
при $i, j, k \in \left\{ x, y, z \right\}, \, i \neq j \neq k$.
Определитель $ \Delta_{xx}$
\begin{gather*}
  \Delta_{xx} =
  -k \cdot S_x \cdot S_{yy} \cdot S_{zz} + k \cdot S_x \cdot S_{yz}^2 +
  k \cdot S_y \cdot S_{xy} \cdot S_{zz} - k \cdot S_y \cdot_{xz} \cdot S_{yz} - \\
  - k \cdot S_z \cdot S_{xy} \cdot S_{yz} + k \cdot S_{z} \cdot S_{xz} \cdot S_{yy} +
  k_x \cdot \left| S \right| \cdot S_{yy} \cdot S_{zz} -
  k_x \cdot \left| S \right| \cdot S_{yz}^2 - \\
  - k_x \cdot S_y^2 \cdot S_{zz} + 2 \cdot k_x \cdot S_y \cdot S_z \cdot S_{yz} -
  k_x \cdot S_z^2 \cdot S_{yy} - k_y \cdot \left| S \right| \cdot S_{yz} \cdot S_{zz} + \\
  + k_y \cdot \left| S \right| \cdot S_{xz} \cdot S_{yz} + k_y \cdot S_x \cdot S_y \cdot S_{zz} -
  k_y \cdot S_x \cdot S_y \cdot S_{yz} - k_y \cdot S_y \cdot S_z \cdot S_{xz} + \\
  + k_y \cdot S_z^2 \cdot S_{xy} + k_z \cdot \left| S \right| \cdot S_{xy} \cdot S_{yz} -
  k_z \cdot \left| S \right| \cdot S_{xz} \cdot S_{yy} - k_z \cdot S_x \cdot S_y \cdot S_{yz} + \\
  + k_z \cdot S_x \cdot S_z \cdot S_{yy} + k_z \cdot S_y^2 \cdot S_{xz} -
  k_z \cdot S_y \cdot S_z \cdot S_{xy}.
\end{gather*}
Определитель $ \Delta_{xy}$
\begin{gather*}
  \Delta_{xy} =
  k \cdot S_x \cdot S_{xy} \cdot S_{zz} - k \cdot S_x \cdot S_{xz} \cdot S_{yz} -
  k \cdot S_y \cdot S_{xx} \cdot S_{zz} + k \cdot S_y \cdot S_{xz}^2 + \\
  + k \cdot S_z \cdot S_{xx} \cdot S_{yz} - k \cdot S_z \cdot S_{xy} \cdot S_{xz} -
  k_x \cdot \left| S \right| \cdot S_{xy} \cdot S_{zz} +
  k_x \cdot \left| S \right| \cdot S_{xz} \cdot S_{yz} + \\
  + k_x \cdot S_x \cdot S_y \cdot S_{zz} - k_x \cdot S_x \cdot S_z \cdot S_{yz} -
  k_x \cdot S_y \cdot S_z \cdot S_{xz} + k_x \cdot S_z^2 \cdot S_{xy} + \\
  + k_y \cdot \left| S \right| \cdot S_{xx} \cdot S_{zz} -
  k_y \cdot \left| S \right| \cdot S_{xz}^2 - k_y \cdot S_x^2 \cdot S_{zz} +
  2 \cdot k_y \cdot S_x \cdot S_z \cdot S_{xz} - \\
  - k_y \cdot S_x \cdot S_z \cdot S_{xz} - k_y \cdot S_z^2 \cdot S_{xx} -
  k_z \cdot \left| S \right| \cdot S_{xx} \cdot S_{yz} +
  k_z \cdot \left| S \right| \cdot S_{xy} \cdot S_{xz} + \\
  + k_z \cdot S_x^2 \cdot S_{yz} - k_z \cdot S_x \cdot S_y \cdot S_{xz} -
  k_z \cdot S_x \cdot S_z \cdot S_{xy} + k_z \cdot S_y \cdot S_z \cdot S_{xx}.
\end{gather*}
Определитель $ \Delta_{xz}$
\begin{gather*}
  \Delta_{xz} =
  -k \cdot S_x \cdot S_{xy} \cdot S_{yz} + k \cdot S_x \cdot S_{xz} \cdot S_{yy} +
  k \cdot S_y \cdot S_{xx} \cdot S_{yz} - \\
  - k \cdot S_y \cdot S_{xy} \cdot S_{xz} - k \cdot S_z \cdot S_{xx} \cdot S_{yy} +
  k \cdot S_z \cdot S_{xy}^2 + S_x \cdot \left| S \right| \cdot S_{xy} \cdot S_{yz} - \\
  - k_x \cdot \left| S \right| \cdot S_{xz} \cdot S_{yy} - k_x \cdot S_x \cdot S_y \cdot S_{yz} +
  k_x \cdot S_x \cdot S_y \cdot S_{yy} + k_x \cdot S_y^2 \cdot S_{xz} - \\
  - k_x \cdot S_y \cdot S_z \cdot S_{xy} - k_y \cdot \left| S \right| \cdot S_{xx} \cdot S_{yz} +
  k_y \cdot \left| S \right| \cdot S_{xy} \cdot S_{xz} + k_y \cdot S_x^2 \cdot S_{yz} - \\
  - k_y \cdot S_x \cdot S_y \cdot S_{xz} - k_y \cdot S_x \cdot S_z \cdot S_{xy} +
  k_y \cdot S_y \cdot S_z \cdot S_{xx} +
  k_z \cdot \left| S \right| \cdot S_{xx} \cdot S_{yy} - \\
  - k_z \cdot \left| S \right| \cdot S_{xy}^2 - k_z \cdot S_x^2 \cdot S_{yy} +
  2 \cdot k_z \cdot S_x \cdot S_y \cdot S_{xy} - k_z \cdot S_y^2 \cdot S_{xx}.
\end{gather*}
Известно, что решениями есть следующие выражения
\begin{equation*}
  b_x = \frac{ \Delta_b}{ \Delta }, \,
  r_{xx} = \frac{ \Delta_{xx}}{ \Delta }, \,
  r_{xy} = \frac{ \Delta_{xy}}{ \Delta }, \,
  r_{xz} = \frac{ \Delta_{xz}}{ \Delta }.
\end{equation*}

Остальные проекции находим аналогичным образом,
приравняв частные производные от $E_x$ и $E_y$ к нулю.

Два множества зачастую не имеют взаимно-однозначного отображения.
Это может приводить к тому,
что два отрезка, расположенные под прямым углом в одном облаке точек,
могут соответствовать трём отрезкам в другом облаке точек,
между соседними парами которых углы по 45 градусов.
