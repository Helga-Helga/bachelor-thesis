\section{Постановка задачи.}
Есть два облака точек: исходное $S$ (sourсe) и целевое $T$ (target).
К точкам исходного множества $ \vec{s} \in S$ применили поворот $R \in \mathbb{R}^{3 \times 3}$
и перемещение $ \vec{b} \in \mathbb{R}^3$,
а также в процессе сканирования появился аддитивный гауссовый шум
с неизвестной дисперсией
\begin{equation}
  \vec{k}_s = R \cdot \vec{s} + \vec{b} + \vec{ \xi }_s, \qquad
  \vec{ \xi }_s \sim N \left( \vec{0}, \sigma^2 \cdot I \right)
\end{equation}
где $k \, : \, S \to T$~---~разметка, то есть функция,
которая сопоставляет каждой точке из исходного множества точку из целевого множества.

Задача состоит в таком выборе матрицы $R$ и вектора $ \vec{b}$,
при которых расстояние между множествами $ \vec{k}_s$ и $R \cdot \vec{s} + \vec{b}$
было бы наименьшим.
В том случае, когда $S$ и $T$ --- конечны,
можно воспользоваться обычным методом наименьших квадратов

\begin{equation}
 E \left( k, R, b \right) =
 \sum \limits_{s \in S}
 \left( \vec{k}_s - R \cdot \vec{s} - \vec{b} \right)^2 \to \min \limits_{k, R, b}.
\end{equation}

Сумма квадратов отклонений между векторами --- это то же самое,
что сумма квадратов отклонений между проекциями по каждой координате.
Найдём, чему равна проекция произведения матрицы $R$ на вектор $s$ на все координаты
\begin{equation*}
  R \cdot \vec{s} =
  \begin{bmatrix}
    r_{xx} & r_{xy} & r_{xz} \\
    r_{yx} & r_{yy} & r_{yz} \\
    r_{zx} & r_{zy} & r_{zz}
  \end{bmatrix} \cdot
  \begin{bmatrix}
    s_x \\
    s_y \\
    s_z
  \end{bmatrix} =
  \begin{bmatrix}
    r_{xx} \cdot s_x + r_{xy} \cdot s_y + r_{xz} \cdot s_z \\
    r_{yx} \cdot s_x + r_{yy} \cdot s_y + r_{yz} \cdot s_z \\
    r_{zx} \cdot s_x + r_{zy} \cdot s_y + r_{zz} \cdot s_z
  \end{bmatrix} =
  \begin{bmatrix}
    \vec{r}_x \cdot \vec{s} \\
    \vec{r}_y \cdot \vec{s} \\
    \vec{r_z} \cdot \vec{s}
  \end{bmatrix}.
\end{equation*}
Тогда можем записать проекции суммы квадратов отклонений на все координаты
\begin{gather*}
  \begin{cases}
    E_x = \sum \limits_{s \in S} \left( \vec{r}_x \cdot \vec{s} + b_x - k_{s_x} \right)^2 \to \min, \\
    E_y = \sum \limits_{s \in S} \left( \vec{r}_y \cdot \vec{s} + b_y - k_{s_y} \right)^2 \to \min, \\
    E_z = \sum \limits_{s \in S} \left( \vec{r}_z \cdot \vec{s} + b_z - k_{s_z} \right)^2 \to \min,
  \end{cases}
\end{gather*}
где $ \vec{r_x} \cdot \vec{s}$ --- проекция $R \cdot \vec{s}$ на ось $x$.

Имеем линейную функциию, которая возводится в квадрат.
Это выпуклая фунция.
Значит, можно взять частные производные по $r_i$ и $x_i$ для всех $i \in \left\{ x, y, z \right\} $
и приравнять их к нулю.
Получим 4 уравнения для каждой координаты.
Запишем для $E_x$, для остальных --- аналогично
\begin{gather*}
  \begin{cases}
    \frac{ \partial E_x}{ \partial b_x} =
    \sum \limits_{s \in S} 2 \left( \vec{r_x} \cdot \vec{s} + b_x - k_{s_x} \right ) = 0, \\
    \frac{ \partial E_x}{ \partial r_{xx}} =
    \sum \limits_{s \in S}
      2 \left( \vec{r_x} \cdot \vec{s} + b_x - k_{s_x} \right ) \cdot s_x = 0, \\
    \frac{ \partial E_x}{ \partial r_{xy}} =
    \sum \limits_{s \in S}
      2 \left( \vec{r_x} \cdot \vec{s} + b_x - k_{s_x} \right ) \cdot s_y = 0, \\
    \frac{ \partial E_x}{ \partial r_{xz}} =
    \sum \limits_{s \in S} 2 \left( \vec{r_x} \cdot \vec{s} + b_x - k_{s_x} \right ) \cdot s_z = 0.
  \end{cases}
\end{gather*}

Решаем первое уравнение относительно $b_x$.
Получаем
\begin{equation*}
  \sum \limits_{s \in S} b_x =
  \sum \limits_{s \in S} \left(k_{s_x} - \vec{r_x} \cdot \vec{s} \right ).
\end{equation*}
Слева получили сумму одинаковых слагаемых
\begin{equation*}
  \left| S \right| \cdot b_x + \vec{r}_x \sum \limits_{s \in S} \vec{s} =
  \sum \limits_{s \in S} k_{s_x}.
\end{equation*}
Распишем скалярное произведение
\begin{equation*}
  \left| S \right| \cdot b_x + \sum \limits_{s \in S} r_{xx} \cdot s_x +
  \sum \limits_{s \in S} r_{xy} \cdot s_y + \sum \limits_{s \in S} r_{xz} \cdot s_z  =
  \sum \limits_{s \in S} k_{s_x}.
\end{equation*}

Решаем остальные уравнения относительно $r_{xi}$ для $i \in \left\{ x, y, z \right\} $.
Видим, что для разных $i$ производная по $r_{xi}$ одинаковая,
потому находим решение для $r_{xx}$, а для остальных решение будет аналогичным
\begin{equation*}
  \vec{r}_x \sum \limits_{s \in S} \vec{s} \cdot s_x =
  \sum \limits_{s \in S} \left( k_{s_x} - b_x \right) \cdot s_x.
\end{equation*}
Распишем скалярное произведение
\begin{equation*}
  r_{xx} \sum \limits_{s \in S} s_x^2 + \sum \limits_{s \in S} b_x \cdot s_x =
  \sum \limits_{s \in S} k_{s_x} \cdot s_x.
\end{equation*}
Получаем систему уравнений
\begin{equation*}
  \begin{cases}
    \left| S \right| \cdot b_x + \sum \limits_{s \in S} r_{xx} \cdot s_x +
    \sum \limits_{s \in S} r_{xy} \cdot s_y + \sum \limits_{s \in S} r_{xz} \cdot s_z  =
    \sum \limits_{s \in S} k_{s_x}, \\
    r_{xx} \sum \limits_{s \in S} s_x^2 + \sum \limits_{s \in S} b_x \cdot s_x =
    \sum \limits_{s \in S} k_{s_x} \cdot s_x, \\
    r_{xy} \sum \limits_{s \in S} s_y \cdot s_x + \sum \limits_{s \in S} b_x \cdot s_x =
    \sum \limits_{s \in S} k_{s_x} \cdot s_y, \\
    r_{xz} \sum \limits_{s \in S} s_z \cdot s_x + \sum \limits_{s \in S} b_x \cdot s_x =
    \sum \limits_{s \in S} k_{s_x} \cdot s_z.
  \end{cases}
\end{equation*}
Запишем её в матричном виде
\begin{equation*}
  \begin{bmatrix}
    \left| S \right| & \sum \limits_{s \in S} s_x & \sum \limits_{s \in S} s_y & \sum \limits_{s \in S} s_z \\
    \sum \limits_{s \in S} s_x & \sum \limits_{s \in S} s_x^2 & 0 & 0 \\
    \sum \limits_{s \in S} s_x & 0 & \sum \limits_{s \in S} s_y \cdot s_x & 0 \\
    \sum \limits_{s \in S} s_x & 0 & 0 & \sum \limits_{s \in S} s_z \cdot s_x
  \end{bmatrix}
  \begin{bmatrix}
    b_x \\
    r_{xx} \\
    r_{xy} \\
    r_{xz}
  \end{bmatrix} =
  \begin{bmatrix}
    \sum \limits_{s\in S} k_{s_x} \\
    \sum \limits_{s\in S} k_{s_x} \cdot s_x \\
    \sum \limits_{s\in S} k_{s_x} \cdot s_y \\
    \sum \limits_{s\in S} k_{s_x} \cdot s_z
  \end{bmatrix}.
\end{equation*}

Два множества зачастую не имеют взаимно-однозначного отображения.
Это может приводить к тому,
что два отрезка, расположенные под прямым углом в одном облаке точек,
могут соответствовать трём отрезкам в другом облаке точек,
между соседними парами которых углы по 45 градусов.
