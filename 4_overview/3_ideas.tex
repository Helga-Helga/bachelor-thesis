\section{Идеи}

\subsection{Симметричный итеративный алгоритм ближайших точек}

Для каждой точки из исходного облака ищем ближайшую точку на целевом облаке.
Вычисляем матрицу поворота и вектор смещения для исходного облака.
Далее для каждой точки из целевого облака находим ближайшую точку на исходном облаке,
оцениваем матрицу поворота и вектор смещения для целевого облака и повторяем все действия снова.

\subsection{Выбор исходного и целевого облаков точек}

В качетсве исходного облака можно брать то, где точек меньше,
а в качестве целевой~---~то, где точек больше.
Таким образом нужно будет найти меньше соответствующих пар точек.

\subsection{Использование инвариантов}

Некоторые характеристики облака точек сохраняются при поворотах и перемещениях,
то есть являются инвариантными относительно них.
Значит, что по этим признакам можно совмещать два облака точек,
не беря во внимание сами трансформации, которые можно будет быстро найти при наличии разметки $k$.
К таким свойствам относятся длины отрезков и площади замкнутых фигур \cite{geometry}.
