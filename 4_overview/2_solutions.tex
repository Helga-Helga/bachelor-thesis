\section{Существующие решения.}

\subsection{Итеративный алгоритм ближайших точек}

Итеративный алгоритм ближайших точек (Iterative Closest Points, ICP) \cite{icp}
состоит из двух чередующихся операций.
Инициализируется алгоритм единичной матрицей $R = I$
и нулевым вектором смещения $ \vec{b} = \vec{0}$.
Первая итерация состоит в поиске разметки с фиксированной трансформацией
\begin{equation}
 \sum \limits_{s \in S}
 \left \Vert \vec{k}_s - R \cdot \vec{s} - \vec{b} \right \Vert^2 \to \min \limits_{k}.
\end{equation}
На следующей итерации происходит поиск поворота и смещения при текущей разметке
\begin{equation}
 \sum \limits_{s \in S}
 \left \Vert \vec{k}_s - R \cdot \vec{s} - \vec{b} \right \Vert^2 \to \min \limits_{R, \vec{b}}.
\end{equation}
Эти два шага повторяются, пока не будет достигнут желаемый результат,
то есть пока расстояние между двумя облаками точек не будет сведено к минимуму.

\subsection{Итеративный алгоритм ближайших точек с нормалями}

Отличием данного алгоритма (Normal ICP) \cite{nicp} является то,
что он рассматривает каждую точку вместе с локальными особенностями поверхности
\begin{equation}
  E \left( k, R, b \right) =
  \sum \limits_{s \in S}
  \alpha_{point} \cdot \left \Vert \vec{k}_s - R \cdot \vec{s} - \vec{b} \right \Vert^2 +
  \alpha_{plane} \cdot
    \left|
      \vec{n}_s^T \cdot \left( \vec{k}_s - R \cdot \vec{s} - \vec{b} \right)
    \right|
  \to \min \limits_{k, R, b},
\end{equation}
где $ \alpha_{point}$ и $ \alpha_{plane}$~---~константы,
а $ \vec{n}_s$~---~нормаль к точке $ \vec{s}$ на исходном облаке.
Для улучшения работы алгоритма убираются
\begin{enumerate}
  \item вершины,
        нормали которых слишком отличаются от нормалей ближайших соседей из целевого облака;
  \item вершины, которые находятся далеко от соседей из целевого облака;
  \item вершины, которые находятся на краю объектов.
\end{enumerate}
