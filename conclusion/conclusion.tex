\chapter*{Висновки}
\addcontentsline{toc}{chapter}{Висновки}

В ході роботи було досліджено метод найменших квадратів.
З деякими змінами цей метод дозволяє розв'язати поставлену
задачу та показує коректність ітеративного алгоритму найближчих точок.

Було з'ясовано, що звичайний метод найменших квадратів
не дає оптимального розв'язку в даному випадку,
адже на шукані параметри накладені нелінійні обмеження:
матриця повороту $R$ має бути ортогональною,
а її визначник повинен дорівнювати одиниці.
Також за допомогою цього методу ми не можемо оцінити оптимальну розмітку $k$,
адже це дискретна функція.

Саме через ці зауваження алгоритм складається з двох кроків,
які виконуються почергово.
На першому кроці алгоритму відбувається пошук найкращої розмітки $k$
при фіксованих $R$ і $\pmb{b}$.
На другому~---~пошук матриці повороту $R$ і вектора зсуву $ \pmb{b}$
при фіксованій розмітці $k$.
Щоб забезпечити ортогональність матриці повороту,
застосовується сингулярний розклад.

Була досліджена відстань Хаусдорфа та $\varepsilon$-сітка
для компактних множин в повному сепарабельному метричному просторі,
завдяки чому можна вибирати точки на неперервних множинах та використовувати
для них ітеративний алгоритм найближчих точок.

Було доведено, що ітеративний алгоритм найближчих точок завжди збігається.
Подальша робота полягає у дослідженні властивостей оцінки параметрів,
отриманої за допомогою алгоритму.
