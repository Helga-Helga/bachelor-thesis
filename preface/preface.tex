\chapter*{Вступ}
\addcontentsline{toc}{chapter}{Вступ}

\textbf{Актуальність роботи}
Оцінка положення камери по хмарам точок (або точковим множинам)
лежить в основі сканування об'єктів за допомогою 3D сканера,
одночасній локалізації і картографування.
Для розв'язання цих задач використовується
ітеративний алгоритм найближчих точок і його модифікації.
У зв'язку з розвитком та компактизацією обчислювальної техніки
з'явилась можливість реалізовувати алгоритм на маленьких компьютерах (наприклад,
бортові компьютери дронів).
Оскільки потрібно, щоб такі пристрої працювали надійно,
треба ретельно вивчити властивості алгоритмів, що використовуються.

\textit{Об'єкт дослідження} ---
методи оцінки параметрів камери.

\textit{Предмет дослідження} ---
алгоритми співставлення точкових множин.

\textit{Метою дослідження}
є аналіз алгоритму співставлення двох точкових множин
та отриманих за його допомогою оцінок невідомих параметрів.

Завдання наступні:
\begin{enumerate}
  \item
    вивчити метод найменших квадратів та властивості оцінок найменших квадратів;
  \item
    ознайомитися з ітеративним алгоритмом найближчих точок,
    що використовується для співставлення двох точкових множин;
  \item
    проаналізувати властивості роботи алгоритму;
  \item
    розробити програмний комплекс для співставлення двох точкових множин
    на основі алгоритму, що досліджується;
  \item
    ознайомитися з поняттям відстані Хаусдорфа для можливості
    застосування алгоритму до неперервних множин.
\end{enumerate}
