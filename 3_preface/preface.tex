\chapter*{Вступление}
\addcontentsline{toc}{chapter}{Вступление}

\textbf{Актуальность работы}
Оценка положения камеры по облакам точек
лежит в основе сканирования объектов с помощью 3D сканера.
Для решения этой задачи используется итеративный алгоритм ближайших точек
и его модификации,
однако они учитывают только расстояния между точками двух множеств.
Взаимосвязь между точками одного облака содержит дополнительную информацию,
которая может повысить качество оценки перемещения камеры.

\textit{Объект исследования} ---
методы оценки параметров камеры.

\textit{Предмет исследования} ---
алгоритмы сопоставления облаков точек.

\textbf{Цель исследования.}
Разработка эффективного алгоритма сопоставления двух облаков точек.

Задания следующие:
\begin{enumerate}
  \item
    иcследовать существующие алгоритмы сопоставления облаков точек;
  \item
    предложить альтернативный алгоритм сопоставления облаков точек.
\end{enumerate}
