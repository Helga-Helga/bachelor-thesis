\chapter{Постановка задачі}

Є дві множини:
вихідна $S \subset \mathbb{R}^3$ та цільова $T \subset \mathbb{R}^3$.
Точки вихідної множини $ \pmb{s} \in S$ повернули за допомогою матриці повороту
\begin{equation*}
  R \in \mathbb{R}^{3 \times 3}, \, \,
  R^T = R^{-1}, \,
  \det{R} = 1
\end{equation*}
та зсунули за допомогою вектора $ \pmb{b} \in \mathbb{R}^3$.
Також в процесі сканування
з'явився адитивний ґаусів шум з невідомою дисперсією
\begin{equation}\label{eq:problem}
  \pmb{k_s} = R \cdot \pmb{s} + \pmb{b} + \pmb{ \xi_s}, \qquad
  \pmb{ \xi_s } \sim N \left( \pmb{0}, \sigma^2 \cdot I \right),
\end{equation}
де $k \, : \, S \to T$~---~розмітка, тобто функція,
яка співставляє кожну точку вихідної множини з точкою з цільової множини.

Задача полягає в такому виборі матриці $R$ та вектора $ \pmb{b}$,
при яких відстань між $ \pmb{k_s}$ та $R \cdot \pmb{s} + \pmb{b}$ для всіх
$ \pmb{s} \in S$ була б найменшою.
